\section{PHƯƠNG PHÁP NGHIÊN CỨU}
Để có được hướng đi đúng đắn cho đề tài lần này, cả nhóm sẽ sử dụng các phương pháp sau để có được cái nhìn chi tiết hơn, đồng thời rút ra điểm cốt lõi để phát triển dự án
\begin{enumerate}
    \item \textbf{Phương pháp phân tích \& tổng hợp :} Phương pháp này tập trung vào việc phân tích và tổng hợp thông tin liên quan đến ứng dụng tìm kiếm nhà trọ. Đây là bước đầu tiên trong quá trình nghiên cứu, trong đó ta thu thập và phân tích các thông tin về nhu cầu của người dùng, tính năng cần có của ứng dụng, các yếu tố liên quan đến nhà trọ (ví dụ: vị trí, giá cả, tiện ích,...) và các ứng dụng tìm kiếm nhà trọ hiện có trên thị trường. Sau đó, các thông tin này được tổng hợp để xác định các yêu cầu cụ thể cho ứng dụng tìm kiếm nhà trọ.
    \item \textbf{Phương pháp thực nghiệm :} Phương pháp thực nghiệm tập trung vào việc triển khai thực hiện các thí nghiệm hoặc các bài toán thực tế để xác minh tính khả thi và hiệu quả của ứng dụng tìm kiếm nhà trọ. Trong trường hợp này, có thể xây dựng một phiên bản mẫu của ứng dụng và thử nghiệm chức năng, hiệu suất và trải nghiệm người dùng của nó. Các phản hồi từ người dùng và dữ liệu thu thập được trong quá trình thực nghiệm sẽ giúp cải thiện và điều chỉnh ứng dụng.
    \item \textbf{Phướng pháp so sánh :} Giúp chỉ ra ưu nhược điểm của ứng dụng tìm kiếm nhà trọ so với các ứng dụng tương tự trên thị trường. Bằng cách nghiên cứu và so sánh các tính năng, giao diện, hiệu suất, độ tin cậy và các yếu tố khác, ta có thể đánh giá được điểm mạnh và điểm yếu của ứng dụng tìm kiếm nhà trọ, từ đó tìm cách cải thiện nó để nâng cao sự cạnh tranh, hoàn thiện của ứng dụng.
    \item \textbf{Phương pháp liệt kê :} Tập trung vào việc liệt kê và phân tích các yếu tố cần có trong ứng dụng tìm kiếm nhà trọ. Các yếu tố này có thể bao gồm: tính năng tìm kiếm bằng cách mô tả, tìm kiếm thông qua các tiện ích, khu vực, thông tin chi tiết về nhà trọ, đánh giá từ người dùng, tiện ích xung quanh, quản lý giao dịch và các tính năng khác. Qua việc liệt kê và phân tích các yếu tố này, ta có thể hiểu rõ hơn về các thành phần và chức năng cần có trong ứng dụng tìm kiếm nhà trọ.
\end{enumerate}