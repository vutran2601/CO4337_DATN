\section{ĐỐI TƯỢNG \& PHẠM VI NGHIÊN CỨU}
Phạm vi nghiên cứu của đề tài nhằm hỗ trợ các đối tượng sử dụng ứng dụng tìm thuê nhà trọ tại Việt Nam trong đó bao gồm các nhóm và cá nhân sau:\\
\textbf{Người thuê nhà:}
\begin{itemize}
    \item Sinh viên: sinh viên đến từ các tính lên các thành phố lớn muốn tìm chỗ thuê trọ  trong các khu vực gần trường đại học hoặc cao đẳng.
    \item Người lao động mới: Các người đi làm mới đến một thành phố có thể là đối tượng quan trọng, đặc biệt là những người cần tìm nhà ổn định trong khoảng thời gian ngắn.
\end{itemize}
\textbf{Chủ nhà trọ:}
\begin{itemize}
    \item Chủ nhà nhỏ: Những người sở hữu một số ít phòng trọ và muốn quảng bá cho việc cho thuê chỗ ở của họ.
    \item Doanh nghiệp quản lý nhà trọ: Các công ty hoặc tổ chức quản lý nhiều căn hộ hoặc phòng trọ có thể quan tâm đến cách tối ưu hóa quá trình cho thuê và giữ chỗ ở , cũng như tham khảo giá cá trị trường để tăng tính cạnh tranh trong lĩnh vực cho thuê nhà trọ.
\end{itemize}
Ứng dụng được phát triển trên nền tảng thiết bị di động với ứng dụng bao gồm giao diện có thể tương tác của các chức năng chính và các yêu cầu phi chức năng sẽ được phát triển đầy đủ. Công cụ tìm kiếm với tính năng cho phép người dùng mô tả thông tin về nhà trọ mà người thuê trọ mong muốn cũng sẽ được nghiên cứu và hiện thực