\chapter*{TÓM TẮT CHƯƠNG}
\textbf{\color{myblue}Chương 1. GIỚI THIỆU ĐỀ TÀI}\\
Phần này sẽ trình bày tổng quan về đề tài, bài toán được đặt ra và cho thấy mục tiêu mà đề tài muốn hướng đến, từ đó làm rõ được ý nghĩa mà đề tài đem lại.
\\\textbf{\color{myblue}Chương 2. KIẾN THỨC NỀN TẢNG}\\
Phần này sẽ trình bày về cơ sở lý thuyết của đề tài. Đây là những lý thuyết quan trọng, làm nền tảng cơ sở để nhóm thực hiện việc triển khai giải pháp đề xuất cho đề tài.
\\\textbf{\color{myblue}Chương 3. KHẢO SÁT CÁC GIẢI PHÁP LIÊN QUAN}\\
Phần này sẽ trình bày một vài giải pháp có liên quan đến đề tài mà nhóm đang hướng đến. Bao gồm giải pháp về lý thuyết mô hình tìm kiếm cũng như giải pháp ứng dụng.
\\\textbf{\color{myblue}Chương 4. GIẢI PHÁP ĐỀ XUẤT}\\
Phần này nhóm sẽ trình bày những bước ban đầu chuẩn bị cho việc hiện thực hệ thống, bao gồm các đặc tả, mô hình, kiến trúc, thiết kế đề xuất.
\\\textbf{\color{myblue}Chương 5. HIỆN THỰC GIẢI PHÁP}\\
Phần này sẽ trình bày quá trình để nhóm hiện thực giải pháp, bao gồm từ việc phát triển mã nguồn, tổ chức và quản lý mã nguồn cũng như trình bày về các kết quả đầu ra đã đạt được.
\\\textbf{\color{myblue}Chương 6. TRIỂN KHAI HỆ THỐNG}\\
Phần này sẽ trình bày quy trình mà nhóm thực hiện triển khai từng phần của hệ thống bao gồm cơ sở dữ liệu, hệ thống phía \textit{back-end}, triển khai mô hình \textit{hybrid search} và xây dựng \textit{file} cài đặt ứng dụng cho người dùng \textit{Android}.
\\\textbf{\color{myblue}Chương 7. KIỂM TRA VÀ ĐÁNH GIÁ GIẢI PHÁP TÌM KIẾM}\\
Phần này sẽ trình bày quy trình mà nhóm thực hiện để kiểm tra và đánh giá giải pháp đề xuất mô hình tìm kiếm kết hợp bao gồm đánh giá về mặt thời gian cũng như độ chính xác.
\\\textbf{\color{myblue}Chương 8. TỔNG KẾT}\\
Phần này sẽ trình bày về các kết quả mà nhóm đã đạt được khi phát triển đề tài, nêu ra những ưu điểm và nhược điểm cũng như đưa ra các định hướng để mở rộng đề tài một cách tốt hơn trong tương lai.