\chapter*{TÓM TẮT ĐỀ TÀI}
\hspace*{1cm} Quá trình đô thị hóa ngày càng gia tăng của Việt Nam đã thu hút nhiều người ngày càng đổ dồn về các đô thị để sinh sống, học tập và làm việc. Do đó nhu cầu về chỗ ở luôn là một vấn đề thiết yếu trong bối cảnh gia tăng dân số đô thị, và nhà trọ là một trong những giải pháp phổ biến nhất vì sự tiện lợi, giá thành rẻ và có thể tìm thấy ở bất cứ đâu. Tuy nhiên đối với những người có nhu cầu tìm kiếm nhà trọ, việc tìm ra một nhà trọ phù hợp không phải là vấn đề đơn giản, thường tốn nhiều thời gian và công sức.\\
\hspace*{1cm} Để đáp ứng được bài toán tìm kiếm nhà trọ, trong nội dung đề tài này nhóm nghiên cứu sẽ tập trung xây dựng giải pháp tìm kiếm nhà trọ trên nền tảng nhà trọ thông minh, bao gồm việc tạo ra một ứng dụng di động giúp kết nối dễ dàng hơn giữa người thuê trọ và người cho thuê, kết hợp với việc nghiên cứu và tích hợp các phương pháp tìm kiếm dữ liệu cũng như hệ thống đề xuất gợi ý người dùng dựa trên các đặc điểm và thói quen tìm kiếm của họ.\\
\hspace*{1cm} Trong suốt giai đoạn vừa qua, nhóm đã từng bước xây dựng ứng dụng qua việc hiện thực cơ sở dữ liệu, phát triển ứng dụng và quan trọng nhất là đã hiện thực mô hình tìm kiếm \textit{hybrid} để thực tiễn hóa giải pháp tìm kiếm tối ưu cho nền tảng nhà trọ thông minh. Tiếp đó, nhóm đã triển khai các phần của ứng dụng trên các máy chủ, đồng thời thực hiện thu thập dữ liệu thực tế từ trang chủ của Nhà Tốt, qua đó cho phép nhóm có thể kiểm tra hiệu quả của mô hình tìm kiếm \textit{hybrid} đối với dữ liệu thực. Từ đó nhóm đã có những đúc kết về đề tài để xây dựng hoàn chỉnh báo cáo, đánh giá kết quả đạt được và phân tích ưu, nhược điểm cũng như định hướng để phát triển đề tài một cách tốt hơn trong tương lai.