\section{CẤU TRÚC MÃ NGUỒN}
\subsection{Cấu trúc mã nguồn front-end}
Trên đây là cấu trúc mã nguồn của phần frontend trong dự án, trong dự án này, cả nhóm sử dụng công nghệ expo go để hiện thực ứng dụng, tại đây sẽ có các thư mục quan trọng như sau:
\begin{itemize}
    \item \textit{app:} chứa source code chính của dự án, như chúng ta có thể quan sát được, những folder được đặt tên trong cặp dấu ( ) như (tabs) hoặc (auth) là định dạng được expo go quy định để áp dụng stack screen và tab screen cho màn hình ứng dụng, \_layout sẽ chứa các màn hình được áp dụng stack screen và tab screen
    \item \textit{asset:} chứa các tài nguyên như hình ảnh, font chữ, các file text,...
    \item \textit{components:} chứa các components được dùng đi dùng lại nhiều lần trong thời gian phát triển app
    \item \textit{server:} chứa các hàm để gọi api push, post, patch, get cho các chức năng trong app như authentication, lấy danh sách nhà, tìm danh sách nhà dựa trên từ khóa,...
    \item \textit{\_layout.tsx:} đây là file để lưu trữ Tab layout có tác dụng cấu trúc tab navigation cho ứng dụng, di chuyển giữa các tab với nhau
    \item \textit{patches:} sử dụng patches để chỉnh sửa các file thư viện trong node modules sao cho trùng khớp với version của nhau để ứng dụng có thể chạy mà không bị gặp lỗi
    \item \textit{eas-json:} cấu hình app để xây dựng ứng dụng dưới dạng file .apk hoặc là link qr code để quét sử dụng
    \item \textit{Thư mục dist:} Thư mục này là kết quả thực thi câu lệnh \textit{npm run build} để tạo ra dự án sản phẩm hoàn thiện có thể sẵn sàng được triển khai trên các nền tảng để sử dụng.
    \item \textit{Thư mục node\_modules:} Thư mục này chứa các gói cài đặt thư viện của bên thứ ba cần thiết để phục vụ cho việc xây dựng các chức năng khác nhau của ứng dụng.
    \item \textit{package-lock.json:} \textit{File} này là một phần quan trọng của dự án \textit{Node.js} sử dụng \textit{npm}. Nó tự động được tạo và cập nhật bởi \textit{npm} khi cài đặt các gói phụ thuộc. \textit{File} này ghi lại chính xác phiên bản của từng gói đã được cài đặt trong dự án, bao gồm các gói phụ thuộc của chúng. Điều này đảm bảo rằng các cài đặt trong tương lai sẽ sử dụng cùng một phiên bản gói, giúp tránh các vấn đề về tương thích và lỗi do thay đổi phiên bản. \textit{File} \textit{package-lock.json} giúp duy trì sự nhất quán và ổn định cho dự án khi làm việc trên nhiều môi trường hoặc khi cộng tác với các nhà phát triển khác.
    \item \textit{package.json:} \textit{File} này chứa thông tin về dự án và các gói phụ thuộc. \textit{File} này định nghĩa các thông tin cơ bản như tên dự án (\textit{name}), phiên bản (\textit{version}), và các \textit{scripts} để thực hiện các tác vụ như xây dựng (\textit{build}), kiểm tra (\textit{test}), và khởi động (\textit{start}) ứng dụng. Các gói phụ thuộc chính (\textit{dependencies}) và phụ thuộc phát triển (\textit{devDependencies}) được liệt kê để quản lý phiên bản và cài đặt. Cấu hình cho \textit{jest} cũng được bao gồm để thiết lập môi trường kiểm thử. \textit{File} này giúp quản lý các công cụ và thư viện cần thiết cho dự án, đảm bảo rằng mọi thành viên trong nhóm sử dụng cùng một phiên bản gói và cấu hình.
\end{itemize}
\subsection{Cấu trúc mã nguồn back-end}
Cấu trúc mã nguồn của \textit{back-end} bao gồm các thư mục và các \textit{file} được mô tả như sau:
\begin{itemize}
    \item \textit{Thư mục dist:} Thư mục này là kết quả thực thi câu lệnh \textit{npm run build} để tạo ra dự án sản phẩm hoàn thiện có thể sẵn sàng được triển khai trên các nền tảng để sử dụng.
    \item \textit{Thư mục node\_modules:} Thư mục này chứa các gói cài đặt thư viện của bên thứ ba cần thiết để phục vụ cho việc xây dựng các chức năng khác nhau của ứng dụng.
    \item \textit{Thư mục prisma:} Thư mục này chứa \textit{file schema.prisma} để ánh xạ cơ sở dữ liệu với cú pháp do \textit{Prisma ORM} định nghĩa để chuyển đổi qua cơ sở dữ liệu thực sự ở \textit{PostgreSQL}. Ngoài ra thư mục này còn chứa \textit{API} để \textit{back-end} có thể làm việc được với cơ sở dữ liệu \textit{PostgreSQL} thông qua lớp trung gian do \textit{Prisma ORM} cung cấp, các \textit{API} này được lưu trữ trong thư mục \textit{client}.
    \item \textit{Thư mục src/domains:} Thư mục này là nơi để hiện thực toàn bộ các chức năng chính cho ứng dụng. Trong đó mỗi chức năng sẽ được tổ chức thành một thư mục riêng, và trong mỗi thư mục sẽ chứa các thư mục \textit{request} và \textit{response} để định nghĩa kiểu cho các yêu cầu \textit{request} và phản hồi \textit{response}. Cùng với đó là các \textit{file <name>.module.ts} để đóng gói các thành phần bên trong thành một chức năng hoàn chỉnh, \textit{file} này quy định về các \textit{controller} và \textit{service} được sử dụng cho chức năng đó, ngoài ra có thể \textit{import} từ \textit{module} khác hoặc xuất ra \textit{service} để \textit{module khác có thể sử dụng}. File \textit{<name>.controller.ts} được dùng để định nghĩa các \textit{endpoint} cho chức năng đó, bao gồm kiểu \textit{HTTP request}, kiểu \textit{request} và \textit{response}, với mỗi \textit{endpoint}, \textit{controller} sẽ gọi tới \textit{method} trong \textit{service} tương ứng. Cuối cùng là \textit{file} \textit{<name>.service.ts}, đây là nơi hiện thực \textit{business logic} dành cho chức năng đó, bao gồm việc xử lý \textit{database}, gọi tới \textit{service} của bên thứ ba...
    \item \textit{Thư mục src/interceptors:} Thư mục này chứa \textit{file logger.interceptor.ts} để ghi lại các thông tin quan trọng về \textit{request} như đường dẫn \textit{endpoint} và phương thức \textit{HTTP}, hỗ trợ trong quá trình \textit{debug} và giám sát ứng dụng bằng cách cung cấp những thông tin chi tiết về các \textit{request} đang được gửi đến. Điều này giúp cho việc phát hiện và sửa lỗi nhanh chóng hơn, đồng thời cũng cung cấp một cách thức để theo dõi hoạt động của hệ thống một cách hiệu quả. \textit{Interceptor} là một phần quan trọng trong việc quản lý và duy trì tính ổn định của ứng dụng \textit{NestJS}.
    \item \textit{Thư mục src/middlewares:} Thư mục này chứa \textit{file error.middleware.ts} để cung cấp một bộ lọc ngoại lệ \textit{(Exception Filter)} quan trọng, giúp ứng dụng xử lý các ngoại lệ và lỗi một cách trực quan. Nó bắt và ghi \textit{log} các ngoại lệ, xác định \textit{status code} phản hồi dựa trên loại ngoại lệ, và cung cấp các thông tin chi tiết như tên, thông điệp lỗi, đường dẫn \textit{request}, và thời gian xảy ra lỗi. Điều này giúp cải thiện quản lý lỗi và dễ dàng theo dõi, giám sát hiệu quả hoạt động của ứng dụng.
    \item \textit{Thư mục src/validation:} Thư mục này chứa \textit{file pipe.validation.ts}  dùng để xử lý và kiểm tra dữ liệu đầu vào. \textit{ValidationPipe} chuyển đổi dữ liệu thành đối tượng dựa trên \textit{metadata} và kiểm tra tính hợp lệ bằng \textit{class-validator}, xử lý lỗi bằng cách ném ra \textit{HttpException} nếu có lỗi. \textit{ParseArrayPipe} biến đổi một mảng giá trị thành mảng các đối tượng, áp dụng kiểm tra tính hợp lệ và ném ra \textit{HttpException} nếu có lỗi. Cả hai loại \textit{pipe} trên đều giúp bảo vệ và xử lý dữ liệu một cách an toàn trước khi chúng được truyền vào các phương thức xử lý chính của ứng dụng \textit{NestJS}.
    \item \textit{Thư mục src/services:} Thư mục này cung cấp các dịch vụ ở bên thứ ba để hỗ trợ cho các chức năng của ứng dụng, bao gồm dịch vụ kết nối với cơ sở dữ liệu, kết nối với dịch vụ đám mây \textit{Cloudinary}, kết nối với mô hình ngữ nghĩa sinh \textit{vector} từ văn bản.
    \item \textit{Thư mục src/utils:} Thư mục này chứa các kiểu lớp dùng chung cho toàn bộ ứng dụng, ví dụ như kiểu \textit{request} dùng cho cho các \textit{API} liên quan đến lấy danh sách, chức năng phân trang, kiểu trả về đối với các phương thức \textit{void}...
    \item \textit{.env:} \textit{File} này chứa các biến môi trường được sử dụng để cấu hình ứng dụng \textit{NestJS}. Các biến bao gồm địa chỉ \textit{URL} kết nối đến cơ sở dữ liệu \textit{PostgreSQL}, cùng với các thông tin nhạy cảm như khóa bí mật \textit{JWT} và thông tin xác thực của dịch vụ \textit{Cloudinary}. Các biến này đảm bảo ứng dụng có thể kết nối và sử dụng các dịch vụ bên ngoài một cách an toàn và hiệu quả. Ngoài ra, còn có biến để cấu hình thời gian hết hạn của \textit{token} truy cập \textit{JWT} và các \textit{URL} khác để tích hợp với các dịch vụ ngoài.
    \item \textit{.eslintrc.js}: \textit{File} này là cấu hình cho \textit{ESLint}, một công cụ phân tích tĩnh mã nguồn để tìm lỗi trong \textit{code}. Cấu hình này được thiết lập để làm việc với \textit{TypeScript} và \textit{Prettier} nhằm đảm bảo mã nguồn tuân theo các quy tắc nhất định và được định dạng nhất quán. Cụ thể, nó sử dụng parser \textit{@typescript-eslint/parser}, các plugin như \textit{@typescript-eslint/eslint-plugin}, và mở rộng các cấu hình đề xuất từ \textit{plugin:@typescript-eslint/recommended} và \textit{plugin/recommended}. Ngoài ra, nó định nghĩa môi trường là \textit{node} và \textit{jest}, và thiết lập một số quy tắc cụ thể để kiểm soát cách \textit{code} được viết và định dạng.
    \item \textit{.gitignore:} Đượcdùng để loại trừ các \textit{file} và thư mục không cần thiết khỏi hệ thống kiểm soát phiên bản \textit{Git}. \textit{File} này bao gồm các thư mục biên dịch như \textit{/dist}, \textit{/node\_modules}, và \textit{/prisma}, cũng như các \textit{file} cấu hình môi trường như \textit{.env}. Nó cũng loại trừ các \textit{file} nhật ký (\textit{logs}, \textit{*.log}) và các thư mục đầu ra kiểm thử (\textit{/coverage}, \textit{/.nyc\_output}). Ngoài ra, \textit{file} này còn loại trừ các \textit{file} và thư mục cấu hình của các IDE và trình soạn thảo, bao gồm \textit{VSCode}, \textit{.idea}, và \textit{.vscode}.
    \item \textit{.pretierrc:} Được sử dụng để định dạng mã nguồn tự động theo các quy tắc đã chỉ định. \textit{File} này thiết lập các quy tắc định dạng như sử dụng dấu nháy đơn (\textit{singleQuote}: \textit{true}), thêm dấu phẩy cuối cùng (\textit{trailingComma}: \textit{all}), độ rộng tab là 4 ký tự (\textit{tabWidth}: 4), không sử dụng tab (\textit{useTabs}: \textit{false}), kết thúc dòng tự động (\textit{endOfLine}: \textit{auto}), sử dụng dấu chấm phẩy (\textit{semi}: \textit{true}), và độ rộng dòng tối đa là 130 ký tự (\textit{printWidth}: 130). Những thiết lập này giúp đảm bảo mã nguồn được định dạng nhất quán trong toàn bộ dự án.
    \item \textit{nest-cli.json:} \textit{File} này thiết lập các thông số cho \textit{Nest CLI}, bao gồm URL \textit{schema} để xác thực cấu hình, bộ sưu tập mặc định là \textit{@nestjs/schematics}, và thư mục nguồn là \textit{src}. Ngoài ra, nó còn chứa tùy chọn biên dịch (\textit{compilerOptions}) để tự động xóa thư mục đầu ra (\textit{deleteOutDir}: \textit{true}) trước khi biên dịch lại. Cấu hình này giúp duy trì tổ chức và nhất quán trong quy trình phát triển ứng dụng \textit{NestJS}.
    \item \textit{package-lock.json:} \textit{File} này là một phần quan trọng của dự án \textit{Node.js} sử dụng \textit{npm}. Nó tự động được tạo và cập nhật bởi \textit{npm} khi cài đặt các gói phụ thuộc. \textit{File} này ghi lại chính xác phiên bản của từng gói đã được cài đặt trong dự án, bao gồm các gói phụ thuộc của chúng. Điều này đảm bảo rằng các cài đặt trong tương lai sẽ sử dụng cùng một phiên bản gói, giúp tránh các vấn đề về tương thích và lỗi do thay đổi phiên bản. \textit{File} \textit{package-lock.json} giúp duy trì sự nhất quán và ổn định cho dự án khi làm việc trên nhiều môi trường hoặc khi cộng tác với các nhà phát triển khác.
    \item \textit{package.json:} \textit{File} này chứa thông tin về dự án và các gói phụ thuộc. \textit{File} này định nghĩa các thông tin cơ bản như tên dự án (\textit{name}), phiên bản (\textit{version}), và các \textit{scripts} để thực hiện các tác vụ như xây dựng (\textit{build}), kiểm tra (\textit{test}), và khởi động (\textit{start}) ứng dụng. Các gói phụ thuộc chính (\textit{dependencies}) và phụ thuộc phát triển (\textit{devDependencies}) được liệt kê để quản lý phiên bản và cài đặt. Cấu hình cho \textit{jest} cũng được bao gồm để thiết lập môi trường kiểm thử. \textit{File} này giúp quản lý các công cụ và thư viện cần thiết cho dự án, đảm bảo rằng mọi thành viên trong nhóm sử dụng cùng một phiên bản gói và cấu hình.
    \item \textit{tsconfig.build.json:} \textit{File} này mở rộng từ cấu hình cơ bản đã được định nghĩa trong \textit{file} \textit{tsconfig.json} (\textit{extends: "./tsconfig.json"}), nghĩa là nó sử dụng các thiết lập đã được định nghĩa trong đó. Đồng thời, nó xác định các thư mục và \textit{file} sẽ bị loại bỏ khỏi quá trình biên dịch (\textit{exclude}). Cụ thể, các thư mục \textit{node\_modules}, \textit{test}, và \textit{dist} sẽ không được bao gồm trong quá trình biên dịch, và tất cả các \textit{file} có tên kết thúc bằng \textit{spec.ts} cũng sẽ được loại bỏ. Cấu hình này giúp đảm bảo rằng \textit{TypeScript} chỉ biên dịch các \textit{file} cần thiết và loại bỏ các \textit{file} không cần thiết trong quá trình phát triển và xây dựng dự án.
    \item \textit{tsconfig.json:} \textit{File} này xác định các tùy chọn biên dịch như \textit{module commonjs}, hỗ trợ \textit{decorator} và \textit{import} mặc định tổng hợp. Nó sử dụng phiên bản \textit{target} là \textit{ES2021} và tạo các \textit{file} bản đồ nguồn để hỗ trợ gỡ lỗi. \textit{File} cũng chỉ định thư mục đầu ra là \textit{./dist} và kích hoạt biên dịch tăng dần để cải thiện hiệu suất. Cấu hình này giúp trong quá trình phát triển và triển khai ứng dụng \textit{TypeScript}, đảm bảo tính nhất quán và hiệu suất trong biên dịch mã nguồn.
    \item \textit{vercel.json:} \textit{File} này được sử dụng để triển khai ứng dụng trên nền tảng \textit{Vercel}. \textit{File} này định nghĩa phiên bản cấu hình là 2 (\textit{"version": 2}), và có hai phần chính là \textit{"builds"} và \textit{"routes"}. Phần \textit{"builds"} chỉ định nguồn của mã nguồn là tệp src/main.ts và sử dụng \textit{module @vercel/node} để xây dựng. Phần \textit{"routes"} quản lý các tuyến đường của ứng dụng, mỗi yêu cầu (bao gồm \textit{GET, POST, PUT, PATCH, DELETE, OPTIONS}) sẽ được định tuyến đến \textit{file src/main.ts}, với cài đặt \textit{header} cho phép \textit{CORS("Access-Control-Allow-Origin: *"}). Cấu hình này giúp đảm bảo ứng dụng của hoạt động mượt mà và có tính linh hoạt trong việc xử lý các yêu cầu khác nhau trên nền tảng \textit{Vercel}.
\end{itemize}
\subsection{Cấu trúc mã nguồn mô hình tìm kiếm hybrid}
\hspace*{1cm}
Cấu trúc mã nguồn của mô hình tìm kiếm \textit{hybrid} bao gồm các thư mục và các \textit{file} được mô tả như sau:
\begin{itemize}
    \item \textit{Thư mục pycache:} Đây là thư mục tự động được \textit{Python} tạo ra để lưu trữ các \textit{file} tin \textit{bytecode} đã được biên dịch (có đuôi \textit{.pyc}). Những \textit{file} này giúp tăng tốc độ khởi động ứng dụng khi chạy lại mã nguồn.
    \item \textit{Thư mục .venv:} Đây là thư mục chứa môi trường ảo của \textit{Python}. Môi trường ảo cho phép tạo một môi trường độc lập để quản lý các gói thư viện mà không ảnh hưởng đến hệ thống \textit{Python} toàn cục. Điều này đặc biệt hữu ích khi cần các phiên bản khác nhau của các gói thư viện cho các dự án khác nhau.
    \item \textit{Thư mục static:} Đây là thư mục thường dùng để lưu trữ các tài nguyên tĩnh như hình ảnh, \textit{file} \textit{CSS}, \textit{file} \textit{JavaScript}, và các tài nguyên khác mà trình duyệt có thể tải trực tiếp mà không cần xử lý từ phía máy chủ. Trong thư mục này có \textit{file} \textit{style.css}, là \textit{file} định kiểu \textit{CSS} để tạo giao diện cho trang \textit{web}.
    \item \textit{Thư mục templates:} Thư mục này chứa các \textit{file} \textit{HTML} dùng làm mẫu để tạo ra các trang \textit{web} động. Thông thường, các \textit{file} \textit{HTML} trong thư mục này sẽ chứa các \textit{placeholder} để được thay thế bằng dữ liệu thực tế từ máy chủ khi trang \textit{web} được tải. Trong thư mục này có \textit{file} \textit{index.html}, là trang chính của ứng dụng \textit{web}.
    \item \textit{\textit{File} .env:} Đây là \textit{file} môi trường \textit{(environment file)} dùng để lưu trữ các biến môi trường như các khóa \textit{API}, thông tin cấu hình cơ sở dữ liệu, và các thông tin bí mật khác. Các biến môi trường này sẽ được đọc vào khi ứng dụng khởi động. Trong\textit{file} này, nhóm định nghĩa kết nối \textit{database} để có thể kết nối từ ứng dụng.
    \item \textit{\textit{File} .gitignore:} \textit{File} này chứa danh sách các \textit{file} và thư mục cần được bỏ qua khi sử dụng \textit{Git} để quản lý phiên bản mã nguồn. Ví dụ: các \textit{file} trong thư mục \textit{\_\_pycache\_\_}, các \textit{file} môi trường \textit{.env}, và thư mục \textit{.venv} thường được liệt kê trong \textit{.gitignore} để không bị theo dõi bởi \textit{Git}.
    \item \textit{app.py:} \textit{File} \textit{app.py} trong dự án này là \textit{file} chính của ứng dụng \textit{web}, được xây dựng bằng \textit{Flask}. Nó định nghĩa ba tuyến đường chính: 
    \begin{itemize}
        \item \textit{/embedding:} Nhận yêu cầu \textit{GET} với dữ liệu văn bản \textit{JSON}, tạo \textit{embedding} từ văn bản và trả về \textit{vector} dưới dạng array và được bọc bởi \textit{JSON}.
        \item \textit{/search:} Nhận yêu cầu \textit{GET} với truy vấn tìm kiếm \textit{JSON}, thực hiện tìm kiếm kết hợp và trả về kết quả dưới dạng \textit{JSON}.
        \item \textit{/}: Hiển thị trang chủ, \textit{render} từ \textit{file} \textit{index.html}.
    \end{itemize}
    Ngoài ra, \textit{file} này còn mở một đường hầm \textit{ngrok} để truy cập ứng dụng từ bên ngoài mạng nội bộ.
    \item \textit{combine\_listing\_information.py:} \textit{File} \textit{combine\_listing\_information.py} chứa một hàm duy nhất là \textit{combine\_listing\_information}. Hàm này nhận vào một danh sách thông tin về nhà thuê và kết hợp các thông tin này thành một chuỗi mô tả đầy đủ. Chuỗi mô tả bao gồm tên nhà thuê, giá thuê, loại, diện tích, số phòng ngủ, số nhà vệ sinh, địa chỉ, mô tả chi tiết, các tiện ích, và đánh giá trung bình. Hàm này giúp tổng hợp và định dạng thông tin nhà thuê thành một văn bản liền mạch duy nhất để sử dụng trong mô hình tìm kiếm ngữ nghĩa.
    \item \textit{embedding.py:} \textit{File} này định nghĩa hàm \textit{model\_generate\_embedding}, sử dụng mô hình \textit{BGEM3FlagModel} để tạo \textit{embedding} từ văn bản. Hàm này nhận vào một chuỗi văn bản và sử dụng mô hình để mã hóa văn bản thành các \textit{vector} nhúng, trả về kết quả dưới dạng một mảng \textit{NumPy}.
    \item \textit{hybrid\_search.py:} \textit{File} này định nghĩa các hàm tìm kiếm cho ứng dụng, bao gồm tìm kiếm toàn văn bản \textit{(FTSearch)}, tìm kiếm ngữ nghĩa \textit{(semantic search)}, và tìm kiếm kết hợp \textit{(hybrid search)}. Các hàm này sử dụng mô hình nhúng văn bản \textit{(embedding)} và mô hình \textit{CrossEncoder} để mã hóa và xếp hạng các kết quả tìm kiếm, đồng thời kết nối với cơ sở dữ liệu \textit{PostgreSQL} để truy xuất thông tin về danh sách nhà trọ.
    \item \textit{requirements.txt:} \textit{File} này liệt kê các gói thư viện \textit{Python} cần thiết cho dự án và phiên bản tương ứng của chúng
\end{itemize}